\documentclass[twocolumn,10pt]{article}
\usepackage[utf8]{inputenc}
\usepackage{graphicx}
\usepackage{amsmath}
\usepackage{amssymb}
\usepackage{geometry}
\usepackage{setspace}
\usepackage{abstract}
\usepackage{titlesec}
\usepackage[backend=biber,style=ieee]{biblatex}
\usepackage[hidelinks]{hyperref}
\PassOptionsToPackage{hyphens}{url} % Pass hyphens option to url package via hyperref

% Page setup
\geometry{margin=1in}
\setstretch{1.2}
\titleformat{\section}{\bfseries\Large}{\thesection}{1em}{}
\titleformat{\subsection}{\bfseries\large}{\thesubsection}{1em}{}
\titleformat{\subsubsection}{\bfseries\normalsize}{\thesubsubsection}{1em}{}

% Add bibliography file
\addbibresource{references.bib}

% Title
\title{State-of-the-Art in Visual Cortical Prostheses: Technological Advances and Future Directions}
\author{
  Marc J. Posthuma\\
  Student Number: 4413105\\
  \texttt{marc.posthuma@ru.nl}\\
  \\
  Supervisor: Prof.\ dr.\ R.J.A.\ van\ Wezel\\
  Department: Neurobiology, Donders Centre for Neuroscience
}
\date{\today}

% Document
\begin{document}
\twocolumn[
    \maketitle
    \begin{onecolabstract}
        \textbf{Abstract:} Visual cortical prostheses represent a revolutionary
        technology within the field of neuro\-prosthetics, aimed at restoring
        vision for individuals with visual impairments through direct neural
        interfaces. This review systematically explores the current
        capabilities, limitations, and future prospects of visual cortical
        prostheses, with a focus on the integration of artificial intelligence
        (AI) to enhance functionality and effectiveness. Key topics include the
        optimization of phosphene patterns, real-time image processing, and
        comparisons with other types of prosthetic devices. The goal is to
        provide a comprehensive overview of the state-of-the-art in visual
        cortical prostheses and propose future research directions.
    \end{onecolabstract}
    \vspace{1cm}
]

\section{Introduction}\label{sec:intro}
\subsection{Background}
The field of neuroprosthetics has witnessed remarkable progress, particularly
with the advent of visual cortical prostheses. These advanced devices offer hope
for restoring vision in individuals with severe visual impairments by
interfacing directly with the brain's visual cortex. Visual cortical prostheses
work by converting visual information from the external environment into neural
signals that the brain can process, effectively bypassing damaged visual
pathways. The core technology involves the generation of phosphenes---perceived
spots of light resulting from electrical stimulation of the visual
cortex~\cite{vandergrintenBiologicallyPlausiblePhosphene2024}. However,
organizing these phosphenes into coherent and interpretable visual patterns
remains a significant challenge~\cite{merabetWhatBlindnessCan2005}.

AI has emerged as a pivotal element in enhancing these prosthetic systems. By
leveraging sophisticated algorithms, AI can optimize stimulation patterns to
create more naturalistic visual experiences for
users~\cite{kriegeskorteDeepNeuralNetworks2015}. AI's role extends to real-time
image processing, allowing the prosthesis to adapt to varying visual
environments and tasks~\cite{marblestoneIntegrationDeepLearning2016}. This
capability is crucial for developing prosthetic systems that closely mimic
natural vision, providing users with more effective and adaptable solutions. The
integration of AI not only improves the functionality of these devices but also
opens new avenues for innovation in how visual information is processed and
perceived~\cite{gallettiCorticalConnectionsArea2001}.

This review aims to provide a comprehensive analysis of visual cortical
prostheses, focusing on the role of AI in advancing these prosthetics. By
examining current capabilities, identifying limitations, and proposing future
research directions, this work seeks to contribute to the ongoing development of
more effective and user-friendly visual prosthetic systems. Combining
technological innovation with neuroscientific insights has the profound
potential to enhance the quality of life for individuals with visual
impairments.

\subsection{Research Question}
This review addresses the following questions:
\begin{itemize}
    \item How is AI leveraged to enhance visual prostheses, particularly in
          optimizing phosphene patterns and real-time image
          processing~\cite{farnumNewVisionVisual2020}?
    \item How do visual cortical prostheses compare with other types of
          prosthetic devices~\cite{graniPersonalizedClosedloopStimulation2022}?
    \item What are the functional differences between AI-enhanced prosthetic
          vision and natural visual processing within the human brain?
\end{itemize}

\section{Technological Advances}\label{sec:tech_advances}
This section covers recent technological advancements in visual cortical
prostheses, including hardware and software innovations.

Recent years have seen significant progress in the development of visual
cortical prostheses, driven by advancements in both hardware and software
systems. These innovations are pivotal in enhancing the functionality,
efficiency, and user experience of these devices.

One major area of advancement is in electrode design and fabrication.
Traditional electrodes have been limited by issues such as biocompatibility,
stability, and the ability to generate precise neural stimulation. Recent
studies have introduced novel materials and fabrication techniques that
significantly improve these aspects. For instance, the development of flexible
and biocompatible electrodes allows for better integration with neural tissue,
reducing the risk of damage and increasing the longevity of the implants
~\cite{xiangFlexibleThreedimensionalElectrode2016}. Furthermore, advances in
microfabrication have enabled the creation of high-density electrode arrays that
can stimulate the visual cortex with greater precision, offering the potential
for more detailed and coherent visual
experiences~\cite{ryuSpatiallyConfinedResponses2020}.

On the software side, the integration of artificial intelligence (AI) has
revolutionized the way visual information is processed and interpreted by
prosthetic systems. AI algorithms, particularly those based on deep learning,
have been employed to optimize stimulation patterns and enhance image processing
capabilities. These algorithms can learn from vast amounts of data to improve
the accuracy and efficiency of visual signal conversion, making the visual
experiences more naturalistic and adaptable to different environments
~\cite{romeniMachineLearningFramework2021}.

Another significant advancement is the implementation of closed-loop systems in
visual cortical prostheses. These systems continuously monitor neural feedback
to adjust stimulation parameters in real-time, thereby enhancing the precision
and effectiveness of visual restoration. Closed-loop systems mimic the natural
feedback mechanisms of the human visual system, providing a more responsive and
user-friendly experience. Recent research has demonstrated the efficacy of these
systems in improving the visual outcomes for users, as they can dynamically
adapt to changes in the environment and the user's neural responses
~\cite{leviEditorialClosedLoopSystems2018}.

Additionally, innovations in wireless technology have enabled the development of
untethered visual cortical prostheses. Wireless systems eliminate the need for
external wires, which not only improves the comfort and aesthetics of the
prostheses but also reduces the risk of infections and mechanical failures.
Advancements in wireless power transfer and data communication have made it
possible to deliver sufficient power and high-fidelity signals to the implants,
ensuring reliable and efficient
operation~\cite{rosenfeldTissueResponseChronically2020}.

The integration of multi-modal sensory input is another promising development in
this field. By incorporating inputs from other senses, such as auditory or
tactile feedback, visual cortical prostheses can provide a more holistic sensory
experience. This multi-modal approach leverages the brain's ability to integrate
information from different sensory modalities, potentially enhancing the overall
perceptual experience and aiding in the interpretation of visual scenes
~\cite{wanArtificialSensoryNeuron2020}.

In conclusion, the technological advancements in electrode design,
microfabrication, artificial intelligence, closed-loop systems, wireless
technology, and multi-modal sensory integration are significantly advancing the
field of visual cortical prostheses. These innovations are crucial for
developing more effective, reliable, and user-friendly devices that can better
restore vision for individuals with severe visual impairments. Continued
research and development in these areas promise to further enhance the
capabilities and accessibility of visual cortical prostheses, paving the way for
their widespread clinical application.

\section{AI Integration}\label{sec:ai_integration}
Discusses the role of AI in processing and enhancing visual data, optimizing
phosphene patterns, and emulating normal brain processing.

\section{Comparison with Natural Systems}\label{sec:comparison}
Explores the differences in processing between prosthetic and natural vision and
how these differences impact user experience.

\section{Limitations and Challenges}\label{sec:limitations}
Details current drawbacks, biocompatibility issues, and areas requiring
improvement in visual cortical prosthesis technology.

\section{Articles Analysis}\label{sec:articles}
Analyzes key articles related to the topic, summarizing their contributions to
understanding visual cortical prostheses.

\subsection{Article 1}
``Towards biologically plausible phosphene simulation for the differentiable
optimization of visual cortical
prostheses''~\cite{vandergrintenBiologicallyPlausiblePhosphene2024}.

\subsection{Article 2}
``New Vision for Visual Prostheses''~\cite{farnumNewVisionVisual2020}.

\subsection{Article 3}
``Toward a personalized closed-loop stimulation of the visual cortex: Advances
and challenges''~\cite{graniPersonalizedClosedloopStimulation2022}.

\section{Objective}\label{sec:objective}
Provides a comprehensive overview of the current state and future potential of
visual cortical prostheses, highlighting technological capabilities, AI
integration, and challenges.

\section{Strategy to Compose Relevant Literature}\label{sec:strategy}
Describes the strategy for compiling and analyzing relevant literature, using
categories like Technological Advances, AI Integration, Comparison with Natural
Systems, and Limitations and Challenges.

\section{Conclusion}\label{sec:conclusion}
Summarizes the key findings of the review and proposes future research
directions.

\printbibliography%

\end{document}
