\documentclass[twocolumn,10pt]{article}
\usepackage[utf8]{inputenc}
\usepackage{graphicx}
\usepackage{amsmath}
\usepackage{amssymb}
\usepackage{geometry}
\usepackage{setspace}
\usepackage{abstract}
\usepackage{titlesec}
\usepackage[backend=biber,style=ieee]{biblatex}
\usepackage[hidelinks]{hyperref}
\PassOptionsToPackage{hyphens}{url} % Pass hyphens option to url package via hyperref
% Set marginparwidth for todonotes
\setlength{\marginparwidth}{2cm} % Adjust the marginparwidth
\usepackage{todonotes} % TODOs

% Page setup
\geometry{margin=1in}
\setstretch{1.2}
\titleformat{\section}{\bfseries\Large}{\thesection}{1em}{}
\titleformat{\subsection}{\bfseries\large}{\thesubsection}{1em}{}
\titleformat{\subsubsection}{\bfseries\normalsize}{\thesubsubsection}{1em}{}

% Add bibliography file
\addbibresource{references.bib}

% Title
\title{State-of-the-Art in Visual Cortical Prostheses: Technological Advances and Future Directions}
\author{
  Marc J. Posthuma\\
  Student Number: 4413105\\
  \texttt{marc.posthuma@ru.nl}\\
  \\
  Radboud University\\
  Supervisor: Prof.\ dr.\ R.J.A.\ van\ Wezel\\
  Department: Neurobiology, Donders Centre for Neuroscience
}
\date{\today}

% Document
\begin{document}

\listoftodos% Remove this line before submission

\twocolumn[
    \maketitle
    \begin{onecolabstract}
        \noindent Visual cortical prostheses represent a revolutionary
        technology within the field of neuro\-prosthetics, aimed at restoring
        vision for individuals with visual impairments through direct neural
        interfaces. This review systematically explores the current
        capabilities, limitations, and future prospects of visual cortical
        prostheses, with a focus on the integration of artificial intelligence
        (AI) to enhance functionality and effectiveness. Key topics include the
        optimization of phosphene patterns, real-time image processing, and
        comparisons with other types of prosthetic devices. The goal is to
        provide a comprehensive overview of the state-of-the-art in visual
        cortical prostheses and propose future research directions.
    \end{onecolabstract}
    \textbf{Keywords:} Visual cortical prostheses, neuroprosthetics, artificial
    intelligence, phosphene patterns, real-time image processing
    \vspace{1cm}
]

\section{Introduction}\label{sec:intro}
\subsection{Background}
The field of neuroprosthetics has witnessed remarkable progress, particularly
with the advent of visual cortical prostheses. These advanced devices offer hope
for restoring vision in individuals with severe visual impairments by
interfacing directly with the brain's visual cortex. Visual cortical prostheses
work by converting visual information from the external environment into neural
signals that the brain can process, effectively bypassing damaged visual
pathways. The core technology involves the generation of phosphenes---perceived
spots of light resulting from electrical stimulation of the visual
cortex~\cite{vandergrintenBiologicallyPlausiblePhosphene2024}. However,
organizing these phosphenes into coherent and interpretable visual patterns
remains a significant challenge~\cite{merabetWhatBlindnessCan2005}.

AI has emerged as a pivotal element in enhancing these prosthetic systems. By
leveraging sophisticated algorithms, AI can optimize stimulation patterns to
create more naturalistic visual experiences for
users~\cite{kriegeskorteDeepNeuralNetworks2015}. AI's role extends to real-time
image processing, allowing the prosthesis to adapt to varying visual
environments and tasks~\cite{marblestoneIntegrationDeepLearning2016}. This
capability is crucial for developing prosthetic systems that closely mimic
natural vision, providing users with more effective and adaptable solutions. The
integration of AI not only improves the functionality of these devices but also
opens new avenues for innovation in how visual information is processed and
perceived~\cite{gallettiCorticalConnectionsArea2001}.

This review aims to provide a comprehensive analysis of visual cortical
prostheses, focusing on the role of AI in advancing these prosthetics. By
examining current capabilities, identifying limitations, and proposing future
research directions, this work seeks to contribute to the ongoing development of
more effective and user-friendly visual prosthetic systems. Combining
technological innovation with neuroscientific insights has the profound
potential to enhance the quality of life for individuals with visual
impairments.

\subsection{Research Question}
This review addresses the following questions:
\begin{itemize}
    \item How is AI leveraged to enhance visual prostheses, particularly in
          optimizing phosphene patterns and real-time image
          processing?
    \item How do visual cortical prostheses compare with other types of
          prosthetic devices?
    \item What are the functional differences between AI-enhanced prosthetic
          vision and natural visual processing within the human brain?
\end{itemize}

\section{Technological Advances}\label{sec:tech_advances}
Recent years have seen significant progress in the development of visual
cortical prostheses, driven by advancements in both hardware and software
systems. These innovations are pivotal in enhancing the functionality,
efficiency, and user experience of these devices.

\subsection{Advancements in Biomaterials and Electrode Design}
One major area of advancement is in electrode design and fabrication.
Traditional electrodes have been limited by issues such as biocompatibility,
stability, and the ability to generate precise neural stimulation. Recent
studies have introduced novel materials and fabrication techniques that
significantly improve these aspects.

\subsubsection*{Conductive Polymers}
The development of flexible and biocompatible electrodes allows for better
integration with neural tissue, reducing the risk of damage and increasing the
longevity of the implants~\cite{xiangFlexibleThreedimensionalElectrode2016}.
Conductive polymers have been instrumental in advancing the design and
functionality of these electrodes.

One notable conductive polymer is PEDOT:PSS (poly (3,4-ethylenedioxythiophene)
polystyrene sulfonate), which has been widely used due to its excellent
electrical conductivity, flexibility, and biocompatibility. PEDOT:PSS coatings
on electrodes improve signal transduction and reduce impedance, which enhances
the quality of neural recordings and
stimulation~\cite{rivnayHighperformanceTransistorsBioelectronics2015}.

Another significant advancement is the use of polyaniline (PANI), a conductive
polymer known for its tunable conductivity and biocompatibility. PANI can be
chemically modified to optimize its electrical properties, making it suitable
for long-term neural interfacing applications. Its use in electrode design has
shown promising results in maintaining stable performance over extended
periods~\cite{almuflehHighlyFlexiblePolyanilineBased2021}.

Polypyrrole (PPy) is another conductive polymer that has been extensively
studied for neural applications. PPy-based electrodes offer a unique combination
of electrical conductivity and mechanical properties that facilitate close
contact with neural tissue. Additionally, PPy can be doped with various
bioactive molecules to promote tissue integration and reduce inflammatory
responses~\cite{zareElectroconductiveMultifunctionalPolypyrrole2021a}.

These advancements in conductive polymers are crucial for the development of
next-generation neural prostheses, offering improved performance,
biocompatibility, and longevity.

\subsubsection*{Nanotechnology}
Advances in microfabrication have enabled the creation of high-density electrode
arrays that can stimulate the visual cortex with greater precision, offering the
potential for more detailed and coherent visual
experiences~\cite{ryuSpatiallyConfinedResponses2020}.

Nanotechnology has introduced several innovative approaches to enhance the
performance and integration of electrodes in visual cortical prostheses. One
such approach is the use of carbon nanotubes (CNTs), which possess exceptional
electrical conductivity and mechanical strength. CNTs can be incorporated into
electrode designs to improve signal transmission and reduce impedance, thereby
enhancing the quality of neural
stimulation~\cite{alegretThreeDimensionalConductiveScaffolds2018}.

Another promising implementation is the use of graphene, a two-dimensional
material known for its outstanding electrical and thermal properties.
Graphene-based electrodes have shown excellent biocompatibility and flexibility,
which are crucial for long-term implantation and stable neural interfaces. The
incorporation of graphene into electrode arrays allows for better integration
with neural tissue and more precise stimulation of the visual
cortex~\cite{luGraphenebasedNeurotechnologiesAdvanced2018}.

Additionally, gold nanostructures have been utilized to enhance electrode
performance. Gold nanoparticles and nanowires can be used to modify the surface
of electrodes, increasing their surface area and improving their electrical
properties. This modification can lead to more efficient charge transfer and
lower stimulation thresholds, resulting in more effective and reliable neural
stimulation~\cite{zareGoldNanostructuresSynthesis2022}.

These nanotechnology-based advancements are paving the way for the development
of more sophisticated and effective visual cortical prostheses, providing users
with improved visual experiences and greater functionality.

\subsubsection*{3D Printing}
Recent advancements in 3D printing has significantly impacted the field of
visual cortical prostheses, particularly in the fabrication of electrodes. 3D
printing offers unparalleled precision and customization capabilities, allowing
for the creation of complex and highly detailed electrode arrays that can be
tailored to the unique anatomical features of individual
patients~\cite{guoImplantableLiquidMetalbased2017}.

This technology enables the production of electrodes that conform to the
specific geometry of the cortical surface, improving contact and integration
with neural tissue, thereby enhancing the accuracy of neural stimulation and
minimizing potential damage to surrounding
tissues~\cite{liuSoftElasticHydrogelbased2019}. Advances in 3D printing
materials, including biocompatible and conductive inks, have improved the
performance and longevity of printed electrodes, providing the necessary
electrical properties while maintaining compatibility with neural tissue.

Additionally, the ability to rapidly prototype and produce electrodes using 3D
printing reduces manufacturing time and cost, facilitating quicker iterations
and refinements in electrode design, which accelerates the development
process~\cite{zhangClimbinginspiredTwiningElectrodes2019}. Recent studies have
demonstrated the potential in creating flexible and biocompatible electrodes,
advancing the state-of-the-art in visual cortical prostheses.

\subsection{Improved Signal Processing and Integration}
Advances in signal processing and integration have been crucial in enhancing the performance and functionality of visual cortical prostheses. High-resolution imaging techniques and sophisticated signal processing algorithms have significantly improved the precision and reliability of these devices.

\subsubsection*{High-Resolution Imaging}
High-resolution imaging techniques have played a pivotal role in the advancement
of visual cortical prostheses by providing detailed maps of the brain's cortical
structures. These techniques allow for more precise placement and targeting of
electrodes, which is essential for effective neural stimulation.

Optical Coherence Tomography (OCT) is one such high-resolution imaging technique
that has been extensively used in neural prosthetics. OCT provides
cross-sectional images of the brain's surface, enabling detailed visualization
of cortical layers and structures. This level of detail allows for the precise
placement of electrodes, improving the effectiveness and safety of neural
stimulation~\cite{xieUseOpticalCoherence2022}.

Functional Magnetic Resonance Imaging (fMRI) provides high-resolution images of
brain activity by detecting changes associated with blood flow. This imaging
technique is valuable for mapping functional areas of the brain, ensuring that
electrodes are placed in regions that will yield the most beneficial outcomes
for the user~\cite{landelleInvestigatingHumanSpinal2021}.

Two-photon microscopy allows for deep imaging of living brain tissue with high
spatial resolution. This technique is particularly useful for observing the
interactions between electrodes and neural tissue over time, providing insights
that can guide the design and optimization of electrode
arrays~\cite{yangIntegratedMicroprismMicroelectrode}.

Furthermore, the development of NIR-II semiconducting polymers has enhanced in
vivo high-resolution imaging capabilities. These polymers offer excellent
penetration depth and spatial resolution, which are critical for accurate
diagnostics and therapeutic applications in neural prosthetics~\cite{wangRecentProgressSecond2023,kangNIRIISemiconductingPolymers2023}.

Deep learning techniques have also been employed to enhance the resolution of
confocal fluorescence microscopy. By using generative models, these techniques
improve the learning ability of imaging systems in the frequency domain,
resulting in significantly higher resolution images that are essential for
detailed neural mapping~\cite{huangEnhancingImageResolution2023}.

These high-resolution imaging techniques are integral to the development of more
precise and effective visual cortical prostheses, facilitating better
integration and performance of these devices.

\subsubsection*{Wireless Communication}
Additionally, innovations in wireless technology have enabled the development of
untethered visual cortical prostheses. Wireless systems eliminate the need for
external wires, which not only improves the comfort and aesthetics of the
prostheses but also reduces the risk of infections and mechanical failures.
Advancements in wireless power transfer and data communication have made it
possible to deliver sufficient power and high-fidelity signals to the implants,
ensuring reliable and efficient
operation~\cite{rosenfeldTissueResponseChronically2020}.

Recent advancements include the development of biphasic quasistatic brain
communication (BP-QBC), a technique that significantly reduces power consumption
while maintaining high data transfer rates. This method leverages
electro-quasistatic signaling to create a low-power, broadband communication
channel between wireless neural implants and external devices, offering a
promising solution for energy-efficient and high-speed data transmission in
neural prosthetics~\cite{chatterjeeBiphasicQuasistaticBrain2023}.

Another innovative approach is the use of feed-forward neural networks to
improve the control of brain-machine interfaces. This simpler neural network
architecture enhances the speed and accuracy of prosthetic control by more
closely mimicking the natural communication pathways between the brain and the
body. Such advancements not only improve the functionality of prosthetic devices
but also enhance their usability for individuals with paralysis or limb
loss~\cite{willseyRealtimeBrainmachineInterface2022}.

Moreover, multidimensional graph neural networks (GNNs) have been employed to
optimize wireless communication policies in neural prosthetics. These networks
use graph-based representations to manage complex data transmission scenarios,
improving the efficiency and reliability of wireless communication between
implants and external devices~\cite{liuMultidimensionalGraphNeural2024}.

These advancements in wireless communication technology are crucial for the
development of next-generation visual cortical prostheses, providing users with
more seamless and reliable neural interfaces.

\subsection{Software and Algorithmic Enhancements}
On the software side, the integration of artificial intelligence (AI) has
revolutionized the way visual information is processed and interpreted by
prosthetic systems. AI algorithms, particularly those based on deep learning,
have been employed to optimize stimulation patterns and enhance image processing
capabilities. These algorithms can learn from vast amounts of data to improve
the accuracy and efficiency of visual signal conversion, making the visual
experiences more naturalistic and adaptable to different
environments~\cite{romeniMachineLearningFramework2021}.

\subsubsection*{Real-Time Data Processing}
[Add relevant text here if available]

\todo[inline]{This is where I left off last time.}

\subsubsection*{Predictive Modeling}
[Add relevant text here if available]

\subsection{Deep Learning Algorithms in Prosthetic Vision}
\subsubsection*{Convolutional Neural Networks (CNNs)}
CNNs are employed to process and classify visual inputs, enhancing the ability
of the prosthetic system to interpret complex visual scenes and improve object
recognition capabilities.

\subsubsection*{Recurrent Neural Networks (RNNs)}
RNNs, including LSTM (Long Short-Term Memory) networks, are used to handle
sequential data, making it possible to maintain temporal continuity in visual
perception and improve the user's ability to track moving objects.

\subsubsection*{Generative Adversarial Networks (GANs)}
GANs are utilized to generate realistic phosphene patterns by training on large
datasets of visual scenes, improving the fidelity and natural appearance of the
visual output.

\subsubsection*{Multidimensional Graph Neural Networks (GNNs)}
GNNs are leveraged to model and interpret complex relationships within
multidimensional data, enhancing the system's ability to understand and process
spatial and relational information. This capability improves the prosthetic
system's performance in recognizing patterns and structures within the visual
input, leading to more accurate and context-aware visual perception.

\subsection{Other Functional Improvements}
Another significant advancement is the implementation of closed-loop systems in
visual cortical prostheses. These systems continuously monitor neural feedback
to adjust stimulation parameters in real-time, thereby enhancing the precision
and effectiveness of visual restoration. Closed-loop systems mimic the natural
feedback mechanisms of the human visual system, providing a more responsive and
user-friendly experience. Recent research has demonstrated the efficacy of these
systems in improving the visual outcomes for users, as they can dynamically
adapt to changes in the environment and the user's neural
responses~\cite{leviEditorialClosedLoopSystems2018}.

\subsubsection*{Closed-Loop Feedback Systems}
Closed-loop systems continuously monitor neural activity and adjust stimulation
parameters in real-time, leading to more precise and individualized visual
restoration.

\subsubsection*{Multi-Modal Sensory Integration}
The integration of multi-modal sensory input is another promising development in
this field. By incorporating inputs from other senses, such as auditory or
tactile feedback, visual cortical prostheses can provide a more holistic sensory
experience. This multi-modal approach leverages the brain's ability to integrate
information from different sensory modalities, potentially enhancing the overall
perceptual experience and aiding in the interpretation of visual
scenes~\cite{wanArtificialSensoryNeuron2020}.

\subsubsection*{Neuroplasticity and Rehabilitation}
[Add relevant text here if available]

\section{AI Integration}\label{sec:ai_integration}
Discusses the role of AI in processing and enhancing visual data, optimizing
phosphene patterns, and emulating normal brain processing.

\section{Comparison with Natural Systems}\label{sec:comparison}
Explores the differences in processing between prosthetic and natural vision and
how these differences impact user experience.

\section{Limitations and Challenges}\label{sec:limitations}
Details current drawbacks, biocompatibility issues, and areas requiring
improvement in visual cortical prosthesis technology.

\section{Objective}\label{sec:objective}
Provides a comprehensive overview of the current state and future potential of
visual cortical prostheses, highlighting technological capabilities, AI
integration, and challenges.

\section{Conclusion}\label{sec:conclusion}
In conclusion, the technological advancements in electrode design,
microfabrication, artificial intelligence, closed-loop systems, wireless
technology, and multi-modal sensory integration are significantly advancing the
field of visual cortical prostheses. These innovations are crucial for
developing more effective, reliable, and user-friendly devices that can better
restore vision for individuals with severe visual impairments. Continued
research and development in these areas promise to further enhance the
capabilities and accessibility of visual cortical prostheses, paving the way for
their widespread clinical application.

\printbibliography%

\end{document}
